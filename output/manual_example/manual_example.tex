\documentclass[]{article}
\usepackage{lmodern}
\usepackage{amssymb,amsmath}
\usepackage{ifxetex,ifluatex}
\usepackage{fixltx2e} % provides \textsubscript
\ifnum 0\ifxetex 1\fi\ifluatex 1\fi=0 % if pdftex
  \usepackage[T1]{fontenc}
  \usepackage[utf8]{inputenc}
\else % if luatex or xelatex
  \ifxetex
    \usepackage{mathspec}
  \else
    \usepackage{fontspec}
  \fi
  \defaultfontfeatures{Ligatures=TeX,Scale=MatchLowercase}
\fi
% use upquote if available, for straight quotes in verbatim environments
\IfFileExists{upquote.sty}{\usepackage{upquote}}{}
% use microtype if available
\IfFileExists{microtype.sty}{%
\usepackage[]{microtype}
\UseMicrotypeSet[protrusion]{basicmath} % disable protrusion for tt fonts
}{}
\PassOptionsToPackage{hyphens}{url} % url is loaded by hyperref
\usepackage[unicode=true]{hyperref}
\hypersetup{
            pdfborder={0 0 0},
            breaklinks=true}
\urlstyle{same}  % don't use monospace font for urls
\usepackage{color}
\usepackage{fancyvrb}
\newcommand{\VerbBar}{|}
\newcommand{\VERB}{\Verb[commandchars=\\\{\}]}
\DefineVerbatimEnvironment{Highlighting}{Verbatim}{commandchars=\\\{\}}
% Add ',fontsize=\small' for more characters per line
\newenvironment{Shaded}{}{}
\newcommand{\KeywordTok}[1]{\textcolor[rgb]{0.00,0.44,0.13}{\textbf{#1}}}
\newcommand{\DataTypeTok}[1]{\textcolor[rgb]{0.56,0.13,0.00}{#1}}
\newcommand{\DecValTok}[1]{\textcolor[rgb]{0.25,0.63,0.44}{#1}}
\newcommand{\BaseNTok}[1]{\textcolor[rgb]{0.25,0.63,0.44}{#1}}
\newcommand{\FloatTok}[1]{\textcolor[rgb]{0.25,0.63,0.44}{#1}}
\newcommand{\ConstantTok}[1]{\textcolor[rgb]{0.53,0.00,0.00}{#1}}
\newcommand{\CharTok}[1]{\textcolor[rgb]{0.25,0.44,0.63}{#1}}
\newcommand{\SpecialCharTok}[1]{\textcolor[rgb]{0.25,0.44,0.63}{#1}}
\newcommand{\StringTok}[1]{\textcolor[rgb]{0.25,0.44,0.63}{#1}}
\newcommand{\VerbatimStringTok}[1]{\textcolor[rgb]{0.25,0.44,0.63}{#1}}
\newcommand{\SpecialStringTok}[1]{\textcolor[rgb]{0.73,0.40,0.53}{#1}}
\newcommand{\ImportTok}[1]{#1}
\newcommand{\CommentTok}[1]{\textcolor[rgb]{0.38,0.63,0.69}{\textit{#1}}}
\newcommand{\DocumentationTok}[1]{\textcolor[rgb]{0.73,0.13,0.13}{\textit{#1}}}
\newcommand{\AnnotationTok}[1]{\textcolor[rgb]{0.38,0.63,0.69}{\textbf{\textit{#1}}}}
\newcommand{\CommentVarTok}[1]{\textcolor[rgb]{0.38,0.63,0.69}{\textbf{\textit{#1}}}}
\newcommand{\OtherTok}[1]{\textcolor[rgb]{0.00,0.44,0.13}{#1}}
\newcommand{\FunctionTok}[1]{\textcolor[rgb]{0.02,0.16,0.49}{#1}}
\newcommand{\VariableTok}[1]{\textcolor[rgb]{0.10,0.09,0.49}{#1}}
\newcommand{\ControlFlowTok}[1]{\textcolor[rgb]{0.00,0.44,0.13}{\textbf{#1}}}
\newcommand{\OperatorTok}[1]{\textcolor[rgb]{0.40,0.40,0.40}{#1}}
\newcommand{\BuiltInTok}[1]{#1}
\newcommand{\ExtensionTok}[1]{#1}
\newcommand{\PreprocessorTok}[1]{\textcolor[rgb]{0.74,0.48,0.00}{#1}}
\newcommand{\AttributeTok}[1]{\textcolor[rgb]{0.49,0.56,0.16}{#1}}
\newcommand{\RegionMarkerTok}[1]{#1}
\newcommand{\InformationTok}[1]{\textcolor[rgb]{0.38,0.63,0.69}{\textbf{\textit{#1}}}}
\newcommand{\WarningTok}[1]{\textcolor[rgb]{0.38,0.63,0.69}{\textbf{\textit{#1}}}}
\newcommand{\AlertTok}[1]{\textcolor[rgb]{1.00,0.00,0.00}{\textbf{#1}}}
\newcommand{\ErrorTok}[1]{\textcolor[rgb]{1.00,0.00,0.00}{\textbf{#1}}}
\newcommand{\NormalTok}[1]{#1}
\usepackage{longtable,booktabs}
% Fix footnotes in tables (requires footnote package)
\IfFileExists{footnote.sty}{\usepackage{footnote}\makesavenoteenv{long table}}{}
\usepackage{graphicx,grffile}
\makeatletter
\def\maxwidth{\ifdim\Gin@nat@width>\linewidth\linewidth\else\Gin@nat@width\fi}
\def\maxheight{\ifdim\Gin@nat@height>\textheight\textheight\else\Gin@nat@height\fi}
\makeatother
% Scale images if necessary, so that they will not overflow the page
% margins by default, and it is still possible to overwrite the defaults
% using explicit options in \includegraphics[width, height, ...]{}
\setkeys{Gin}{width=\maxwidth,height=\maxheight,keepaspectratio}
\usepackage[normalem]{ulem}
% avoid problems with \sout in headers with hyperref:
\pdfstringdefDisableCommands{\renewcommand{\sout}{}}
\IfFileExists{parskip.sty}{%
\usepackage{parskip}
}{% else
\setlength{\parindent}{0pt}
\setlength{\parskip}{6pt plus 2pt minus 1pt}
}
\setlength{\emergencystretch}{3em}  % prevent overfull lines
\providecommand{\tightlist}{%
  \setlength{\itemsep}{0pt}\setlength{\parskip}{0pt}}
\setcounter{secnumdepth}{0}
% Redefines (sub)paragraphs to behave more like sections
\ifx\paragraph\undefined\else
\let\oldparagraph\paragraph
\renewcommand{\paragraph}[1]{\oldparagraph{#1}\mbox{}}
\fi
\ifx\subparagraph\undefined\else
\let\oldsubparagraph\subparagraph
\renewcommand{\subparagraph}[1]{\oldsubparagraph{#1}\mbox{}}
\fi

% set default figure placement to htbp
\makeatletter
\def\fps@figure{htbp}
\makeatother


\date{mayo  1, 2020}

\begin{document}

\section{Manual Name}\label{manual-name}

\begin{center}\rule{0.5\linewidth}{\linethickness}\end{center}

\subsection{Subname}\label{subname}

\subsubsection{Version}\label{version}

\paragraph{1.0.0}\label{section}

\begin{center}\rule{0.5\linewidth}{\linethickness}\end{center}

\section{Content}\label{content}

\begin{enumerate}
\def\labelenumi{\arabic{enumi}.}
\tightlist
\item
  \protect\hyperlink{1}{Introduction}
\item
  \protect\hyperlink{2}{Second Item}
\item
  \protect\hyperlink{3}{Third Item}

  \begin{enumerate}
  \def\labelenumii{\arabic{enumii}.}
  \tightlist
  \item
    \protect\hyperlink{4}{Another Item}
  \item
    \protect\hyperlink{5}{Other One Item}
  \end{enumerate}
\item
  \protect\hyperlink{6}{Sixth Item}
\end{enumerate}

\begin{center}\rule{0.5\linewidth}{\linethickness}\end{center}

\begin{itemize}
\tightlist
\item
  \protect\hyperlink{1}{1. Introduction}
\item
  \protect\hyperlink{2}{2. Second Item}
\item
  \protect\hyperlink{3}{3. Third Item}
\item
  \protect\hyperlink{4}{3.1. Another Item}
\item
  \protect\hyperlink{5}{3.2. Other One Item}
\item
  \protect\hyperlink{6}{4. Sixth Item}
\end{itemize}

\section{Introduction \{\#1\}}\label{introduction-1}

Hello World.

\begin{center}\rule{0.5\linewidth}{\linethickness}\end{center}

\subsection{Headings}\label{headings}

\section{Heading 1}\label{heading-1}

\subsection{Heading 2}\label{heading-2}

\subsubsection{Heading 3}\label{heading-3}

\paragraph{Heading 4}\label{heading-4}

\subparagraph{Heading 5}\label{heading-5}

Heading 6

\begin{center}\rule{0.5\linewidth}{\linethickness}\end{center}

\subsection{New lines}\label{new-lines}

To write a new line in the same paragraph just add two spaces at the end
of the first line.\\
Like this.

To write a new line for a new paragraph, just place an empty line
between the text lines.

Like this.

\begin{center}\rule{0.5\linewidth}{\linethickness}\end{center}

\subsection{Italic, Bold and
strikethrough}\label{italic-bold-and-strikethrough}

This is how to \emph{use italic} text.\\
This is how to \textbf{use bold} text.\\
This is how to \textbf{\emph{use both}} at same time.\\
This is how to \sout{use strikethrough} text.

\begin{center}\rule{0.5\linewidth}{\linethickness}\end{center}

\subsection{Blockquotes}\label{blockquotes}

Simple:

\begin{quote}
This is a blockquote text\\
with multiples lines
\end{quote}

Nested one:

\begin{quote}
And this is a blockquote \textgreater{} with nested\\
\textgreater{}\textgreater{} lines also
\end{quote}

\begin{center}\rule{0.5\linewidth}{\linethickness}\end{center}

\subsection{Links}\label{links}

This is a \href{https://www.google.com}{link to Google}

You can write simple URL link by itself (without text):
\url{https://www.google.com}

You can write raw (no link) URL:\\
\texttt{https://www.google.com}

\begin{center}\rule{0.5\linewidth}{\linethickness}\end{center}

\subsection{Images}\label{images}

\begin{figure}
\centering
\includegraphics{https://www.google.com/images/branding/googlelogo/1x/googlelogo_color_272x92dp.png}
\caption{image alt text}
\end{figure}

\begin{center}\rule{0.5\linewidth}{\linethickness}\end{center}

\subsection{Code and Highlight Sintax}\label{code-and-highlight-sintax}

This is \texttt{inline\ code} inside a text.

This is C formated multi-line code:

\begin{Shaded}
\begin{Highlighting}[]
\PreprocessorTok{#include }\ImportTok{<stdio.h>}

\DataTypeTok{int}\NormalTok{ main(}\DataTypeTok{void}\NormalTok{)}
\NormalTok{\{}
\NormalTok{    printf(}\StringTok{"Hello World}\SpecialCharTok{\textbackslash{}n}\StringTok{."}\NormalTok{);}
    \ControlFlowTok{return} \DecValTok{0}\NormalTok{;}
\NormalTok{\}}
\end{Highlighting}
\end{Shaded}

This is Python formated multi-line code:

\begin{Shaded}
\begin{Highlighting}[]
\BuiltInTok{print}\NormalTok{(}\StringTok{"Hello World"}\NormalTok{)}
\end{Highlighting}
\end{Shaded}

This is JSON formated multi-line code:

\begin{Shaded}
\begin{Highlighting}[]
\FunctionTok{\{}
  \DataTypeTok{"firstName"}\FunctionTok{:} \StringTok{"John"}\FunctionTok{,}
  \DataTypeTok{"lastName"}\FunctionTok{:} \StringTok{"Doe"}\FunctionTok{,}
  \DataTypeTok{"age"}\FunctionTok{:} \DecValTok{30}
\FunctionTok{\}}
\end{Highlighting}
\end{Shaded}

\begin{center}\rule{0.5\linewidth}{\linethickness}\end{center}

\subsection{Tables}\label{tables}

\begin{longtable}[]{@{}lcr@{}}
\toprule
Syntax & Description & Test Text\tabularnewline
\midrule
\endhead
Header & Title & Here's this\tabularnewline
Paragraph & Text & And more\tabularnewline
\bottomrule
\end{longtable}

 Align cells to left by using :---\\
Align cells to center by using :----:\\
Align cell to right by using ---:

\begin{center}\rule{0.5\linewidth}{\linethickness}\end{center}

\subsection{Definition List}\label{definition-list}

\begin{description}
\tightlist
\item[First Term]
This is the definition of the first term.
\item[Second Term]
This is one definition of the second term.

This is another definition of the second term.
\end{description}

\begin{center}\rule{0.5\linewidth}{\linethickness}\end{center}

\subsection{Checkboxs List}\label{checkboxs-list}

\begin{itemize}
\tightlist
\item
  {[} {]} First item
\item
  {[}x{]} Second item
\item
  {[} {]} Third item
\end{itemize}

\subsection{Emojis}\label{emojis}

Emojis can also be used 😀👍\\
Check: \url{https://emojipedia.org}

\subsection{Footnotes}\label{footnotes}

Let's try a footnote. \href{This\%20is\%20a\%20footnote.}{¹}\\
Some normal text here.\\
And another footnote. \href{Second\%20footnote}{²}

\end{document}
